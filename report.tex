\documentclass[10pt,a4paper]{article}

\usepackage[margin=1.5cm]{geometry}
\usepackage{setspace}
\setstretch{0.96}
\usepackage{parskip}
\setlength{\parskip}{4pt}
\setlength{\parindent}{0pt}
\usepackage{amsmath,amssymb}
\usepackage{graphicx}
\usepackage{xcolor}
\usepackage{hyperref}
\usepackage{multicol}

\hypersetup{
    colorlinks=true,
    linkcolor=blue!40!black
}

\title{\vspace{-1cm}Electric vehicle modelisation exercise\vspace{-0.4cm}}
\date{}
\author{Ranbawale R., Zeinstra M., Donnenfeld T.}

\begin{document}
\maketitle
\vspace{-1cm}


\section*{0. Assignment Assumptions}

\begin{multicols}{3}

\textbf{Vehicle params}
\begin{itemize}\setlength{\itemsep}{1pt}
    \item Mass: 1800 kg
    \item Wheel diameter: 0.66 m
    \item Aero: $C_d=0.30$, $A=2.35$ m$^2$
    \item Rolling Resistance: $C_{rr}=0.010$
    \item Consumption: 17.5 kWh/100 km
\end{itemize}

\columnbreak

\textbf{Targets}
\phantomsection
\label{sec:targets}
\begin{itemize}\setlength{\itemsep}{1pt}
    \item 0--100 km/h $\leq 8$ s
    \item Top speed $\geq 160$ km/h
    \item WLTP range $\approx 380$ km
    \item 10--80\% charge $\leq 28$ min
\end{itemize}

\columnbreak

\textbf{Battery}
\begin{itemize}\setlength{\itemsep}{1pt}
    \item LFP 90/120 Ah, NMC 120 Ah
    \item Usable SOC: 0.95
    \item Charge efficiency: 0.90
    \item Drivetrain efficiency: 0.95
\end{itemize}

\end{multicols}

\vspace{-2mm}


\section*{1. Motor and Gearing Selection}

A permanent-magnet synchronous motor (PMSM) was chosen for its high torque density which makes it the standard choice for modern BEVs.

\textbf{Motor specification}
\begin{itemize}\setlength{\itemsep}{1pt}
    \item Peak power: 150 kW \quad Continuous: 75 kW
    \item Peak torque: 300 Nm \quad Continuous: 150 Nm
    \item Max speed: 13,000 rpm
\end{itemize}

\textbf{Gearing}  

Using single gear.

\[
i = 9.5 : 1
\]

\textbf{Launch wheel torque}
\[
T_{w,0} = 300 \times 9.5 \times 0.95 = 2708\ \text{Nm}
\]
\[
a_0 = \frac{2708/0.33}{1800} = 4.56\ \text{m/s}^2 \approx 0.46g
\]

\textbf{Torque at 100 km/h (constant-power region)}
\[
T_{100} = \frac{150\,000}{799} = 188\ \text{Nm}
\]
\[
T_{w,100} = 188 \times 9.5 \times 0.95 = 1693\ \text{Nm}
\]


\section*{2. Battery Sizing}

NMC 120 Ah was selected due to its higher energy and power density compared to LFP.

\textbf{Configuration}
\[
160s1p
\]
\[
V_{\text{nom}} = 160 \times 3.65 = 584\ \text{V}
\]

\textbf{Energy}
\[
E_{\text{gross}} = 584 \times 120/1000 = 70.1\ \text{kWh}
\qquad
E_{\text{usable}} = 0.95\,E_{\text{gross}} = 66.6\ \text{kWh}
\]

\textbf{Range estimation}  
The WLTP-equivalent consumption level used here is coherent with typical values for BEVs.

\[
\frac{66.6}{17.5}\times 100 = 380\ \text{km}
\]


\section*{3. Performance Verification}

\subsection*{Top speed}

We verify that the motor can sustain 160 km/h on level road. Speed conversion:
\[
v = 160\,\text{km/h} = \frac{160}{3.6} = 44.44\ \text{m/s}
\]

Aerodynamic drag and rolling resistance:
\[
\begin{aligned}
F_d   &= \tfrac{1}{2}\rho C_d A v^2 
      = \tfrac{1}{2}\times 1.2 \times 0.30 \times 2.35 \times 44.44^2
      = 838\ \text{N}, \\[3pt]
F_{rr} &= mgC_{rr}
       = 1800 \times 9.81 \times 0.010
       = 177\ \text{N}, \\[6pt]
F_{\text{res}} &= F_d + F_{rr} 
               = 838 + 177 
               = 1015\ \text{N}.
\end{aligned}
\]

Wheel and motor power:
\[
\begin{aligned}
P_{\text{wheel}} &= F_{\text{res}}\, v 
                 = 1015 \times 44.44
                 = 45.1\ \text{kW}, \\[6pt]
P_m &= \frac{P_{\text{wheel}}}{\eta}
    = \frac{45.1}{0.9}
    \approx 50\ \text{kW}.
\end{aligned}
\]

Our motor has peak power of 150kWh which is well above the computed necessary power : 50kWh.


\subsection*{Gradeability}

We check that the vehicle can maintain 100 km/h on a 6\% slope (maybe not targeting super high performance).

Speed conversion:
\[
v = 100\,\text{km/h} = \frac{100}{3.6} = 27.78\ \text{m/s}
\]

Forces at 100 km/h:
\[
\begin{aligned}
F_d &= \tfrac{1}{2}\rho C_d A v^2
    = \tfrac{1}{2}\times 1.2 \times 0.30 \times 2.35 \times 27.78^2
    = 326\ \text{N}, \\[3pt]
F_{rr} &= mgC_{rr}
       = 1800 \times 9.81 \times 0.010
       = 177\ \text{N}, \\[3pt]
F_g &= mg \cdot \text{grade}
    = 1800 \times 9.81 \times 0.06
    = 1060\ \text{N}, \\[6pt]
F_{\text{res}} &= F_d + F_{rr} + F_g 
               = 326 + 177 + 1060 
               = 1563\ \text{N}.
\end{aligned}
\]

Wheel and motor power on the slope:
\[
\begin{aligned}
P_{\text{wheel}} &= F_{\text{res}}\,v
                 = 1563 \times 27.78
                 = 43.4\ \text{kW}, \\[6pt]
P_m &= \frac{P_{\text{wheel}}}{\eta}
    = \frac{43.4}{0.9}
    = 48.2\ \text{kW}.
\end{aligned}
\]

Since the continuous motor rating is 75 kW, the vehicle can sustain 100 km/h on a 6\% grade.



\section*{4. Charging Capability}

The charge window is 10--80\%.

Lets calculate the added energy:
\[
\Delta E = 0.7 \times 70.1 = 49.1\ \text{kWh}
\]

\[
P_{\text{batt}} = 49.1/0.467 = 105\ \text{kW}
\qquad
P_{\text{plug}} = 105/0.9 = 117\ \text{kW}
\]

\textbf{C-rate}
\[
I = \frac{105000}{584} = 180\ \text{A}
\qquad
C = 180/120 = 1.5
\]

Peak capability required: 2--2.5C.

\newpage
\section*{Conclusion}

In the end, using : 
\begin{itemize}\setlength{\itemsep}{0pt}
  \item 150 kW PMSM motor
  \item 9.5:1 single reduction gear 
  \item 70 kWh NMC battery 
\end{itemize}

Meet all design objectives defined in the \hyperref[sec:targets]{Targets section}.

\end{document}

