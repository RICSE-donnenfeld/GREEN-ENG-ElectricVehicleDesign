\documentclass[10pt,a4paper]{article}

\usepackage[margin=1.5cm]{geometry}
\usepackage{setspace}
\setstretch{0.96}
\usepackage{parskip}
\setlength{\parskip}{4pt}
\setlength{\parindent}{0pt}
\usepackage{amsmath,amssymb}
\usepackage{graphicx}
\usepackage{hyperref}
\usepackage{multicol}

\title{\vspace{-1cm}Battery-Electric Vehicle Sizing Exercise\vspace{-0.4cm}}
\date{}
\author{Zeinstra M., Donnenfeld T.}

\begin{document}
\maketitle
\vspace{-1cm}


\section*{0. Assignment Assumptions}


\begin{multicols}{3}

\textbf{Vehicle params}
\begin{itemize}\setlength{\itemsep}{1pt}
    \item Curb mass: 1800 kg
    \item Wheel diameter: 0.66 m
    \item Aero: $C_d=0.30$, $A=2.35$ m$^2$
    \item Rolling: $C_{rr}=0.010$
    \item Consumption: 17.5 kWh/100 km
\end{itemize}

\columnbreak

\textbf{Targets}
\begin{itemize}\setlength{\itemsep}{1pt}
    \item 0--100 km/h $\leq 8$ s
    \item Top speed $\geq 160$ km/h
    \item WLTP range $\approx 380$ km
    \item 10--80\% charge $\leq 28$ min
\end{itemize}

\columnbreak

\textbf{Battery}
\begin{itemize}\setlength{\itemsep}{1pt}
    \item LFP 90/120 Ah, NMC 120 Ah
    \item Usable SOC: 0.95
    \item Charge efficiency: 0.90
    \item Drivetrain efficiency: 0.95
\end{itemize}

\end{multicols}

\vspace{-2mm}


\section*{1. Motor and Gearing Selection}


A permanent-magnet synchronous motor (PMSM) was chosen for its high torque density and efficiency.  
\textit{Reference to course PDF:} Section 3 (“Electric Motors for EVs”) states that PMSMs offer the best compromise of torque density, efficiency, and compactness for BEVs, explaining their dominant use in the industry.

\textbf{Motor specification}
\begin{itemize}\setlength{\itemsep}{1pt}
    \item Peak power: 150 kW \quad Continuous: 75 kW
    \item Peak torque: 300 Nm \quad Continuous: 150 Nm
    \item Max speed: 13,000 rpm
\end{itemize}

\textbf{Gearing (single reduction)}  
\textit{Course reference:} Section 4 (“Powertrain Architecture”) highlights that EVs typically use a single-speed due to wide motor efficiency range.

\[
i = 9.5 : 1
\]

\textbf{Launch wheel torque}
\[
T_{w,0} = 300 \times 9.5 \times 0.95 = 2708\ \text{Nm}
\]
\[
a_0 = \frac{2708/0.33}{1800} = 4.56\ \text{m/s}^2 \approx 0.46g
\]

\textbf{Torque at 100 km/h (constant-power region)}
\[
T_{100} = \frac{150\,000}{799} = 188\ \text{Nm}
\]
\[
T_{w,100} = 188 \times 9.5 \times 0.95 = 1693\ \text{Nm}
\]


\section*{2. Battery Pack Sizing}


NMC 120 Ah was selected due to higher energy density.  
\textit{Reference to course PDF:} Section 5 (“Battery Cell Technologies”) shows that NMC offers significantly higher gravimetric/volumetric energy density compared with LFP, making it more appropriate for range-sensitive designs.

\textbf{Pack configuration}
\[
160s1p
\]
\[
V_{\text{nom}} = 160 \times 3.65 = 584\ \text{V}
\]

\textbf{Pack energy}
\[
E_{\text{gross}} = 584 \times 120/1000 = 70.1\ \text{kWh}
\qquad
E_{\text{usable}} = 0.95\,E_{\text{gross}} = 66.6\ \text{kWh}
\]

\textbf{Range estimation}  
\textit{Reference:} WLTP consumption levels from Section 6 (“Driving Cycles and Energy Consumption”) are used.

\[
\frac{66.6}{17.5}\times 100 = 380\ \text{km}
\]


\section*{3. Performance Verification}


\subsection*{Top speed (160 km/h)}

\[
F_d = 0.5\,\rho\,C_d A v^2 = 838\ \text{N}
\qquad
F_{rr} = 177\ \text{N}
\]
\[
P_m = \frac{(838 + 177)\,44.44}{0.9} \approx 50\ \text{kW}
\]

\textit{Reference:} Section 7 (“Longitudinal Vehicle Dynamics”) describes the exact drag/rolling model used to determine power vs speed.

\subsection*{Gradeability (6\% @ 100 km/h)}
\[
F_d=326,\quad F_{rr}=177,\quad F_g=1060\ \text{N}
\]
\[
P_m=\frac{1563 \times 27.78}{0.9}=48.2\ \text{kW}
\]

\textit{Reference:} Grade force $F_g = mg \times \text{grade}$ follows Section 7.3 of the course PDF.


\section*{4. Charging Capability}


\textit{Reference:} Section 5.2 (“Charge/Discharge Characteristics”) defines the C-rate convention and the SOC windows used here.

Energy added during 10--80\%:
\[
\Delta E = 0.7 \times 70.1 = 49.1\ \text{kWh}
\]

\[
P_{\text{batt}} = 49.1/0.467 = 105\ \text{kW}
\qquad
P_{\text{plug}} = 105/0.9 = 117\ \text{kW}
\]

\textbf{C-rate}
\[
I = \frac{105000}{584} = 180\ \text{A}
\qquad
C = 180/120 = 1.5
\]

Peak charging capability required: 2--2.5C.


\section*{Conclusion}


The selected 150 kW PMSM motor, 9.5:1 single reduction gear, and 70 kWh NMC battery pack meet all design objectives: acceleration, 160 km/h top speed, WLTP-equivalent 380 km range, 6\% gradeability at 100 km/h, and fast-charging 10--80\% in 28 minutes at an average plug power of 117 kW.  

\end{document}

